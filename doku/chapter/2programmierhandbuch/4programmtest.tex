%%%%%%%%%%%%%%%%%%%%%%%%%%%%%%%%%%%%%%%%%%%%%%%%%%%%%%%%%%%%%%%%%%%%%%%
\section{Programmtest}
\label{sec:programmierhandbuch:test}

Zur Beschreibung eines Programmtests geh�rt eine Auflistung der Testeingabedaten, der erwarteten und der errechneten Ergebnisdaten. Bei Abweichungen sind Begr�ndungen anzugeben und das Programm entsprechend zu beurteilen (z.B. "Eine Ungenauigkeit von 0.005 \% ergibt sich aus dem numerischen Darstellungsbereich des Datentyps XYZ und ist f�r die erwarteten Werte vertretbar."). Grundlage der Auswahl der Testdaten ist der White Box-Test, der gew�hrleistet, da� alle Programmteile durchlaufen werden. Die Auswahl der verwendeten Testdaten ist zu begr�nden, z.B. Indexwerte 99, 100 und 101, um bei einem Array von 100 Werten die Grenz�berschreitung zu pr�fen.