%%%%%%%%%%%%%%%%%%%%%%%%%%%%%%%%%%%%%%%%%%%%%%%%%%%%%%%%%%%%%%%%%%%%%%%
\section{Problemanalyse und Realisation}
\label{sec:programmierhandbuch:analyse_realisation}
%Es soll dargestellt werden, mit welcher Konzeption bzw. nach welchem Algorithmus das Programm umgesetzt worden ist. Es ist zu erkl�ren, warum etwas so und nicht anders gel�st wurde. Dieser Abschnitt ist v�llig unabh�ngig von der verwendeten Programmiersprache zu halten. Hier finden sich Eure �berlegungen, die Ihr vor dem Schreiben der 1. Codezeile gemacht habt, also z.B. ob Datenstruktur1 oder Datenstruktur2 besser zum gegebenen Problem pa�t und warum dies so ist, welche Algorithmen zur Probleml�sung m�glich sind und welcher davon warum umgesetzt wurde usw.

In diesem Abschnitt werden die drei wesentlichen Bestandteile des Plug-In beschrieben. So werden Struktur des Plug-In, ansprechen der Ausgabeger�te und generieren der Blending-Maske analysiert, sowie die implementierung erkl�rt.

\subsection{Struktur des Compiz Plug-In}
\todo{}
\subsubsection{Analyse}

\subsubsection{Realisation}



\subsection{Ansprechen der Ausgabeger�te}
Hinter diesem Punkt verbirgt sich die Beschreibung der verwendeten Methoden um die Bilddaten zu manipulieren und eine ansonsten normale Funktionalit�t des Betriebssystems zu gew�hrleisten.

\subsubsection{Analyse}
\todo{}
\subsubsection{Realisation}



\subsection{Generieren der Blending-Maske}
Nachdem im vorherigen Abschnitt beschrieben wurde, wie die erzeugte Maske mit der eigentlichen Benutzeroberfl�che zusammengef�gt wird, beschreibt dieser Abschnitt wie die Maske erstellt wird.

\subsubsection{Analyse}


\subsubsection{Realisation}

%(- parsen der konfiguration)