%%%%%%%%%%%%%%%%%%%%%%%%%%%%%%%%%%%%%%%%%%%%%%%%%%%%%%%%%%%%%%%%%%%%%%%
\section{Beschreibung grundlegender Datenstrukturen}
\label{sec:programmierhandbuch:datenstrukturen}

%Hier werden die Datenorganisation und die Datenstrukturen zur Probleml�sung beschrieben. Dazu z�hlt der Aufbau, die Gr��e der einzelnen Komponenten und speziell bei dynamischen Strukturen die Verzeigerung untereinander. In diesem Zusammenhang ist auch der Aufbau der vom Programm benutzten Datendateien zu beschreiben.
\subsection{Konfigurationsdatei und deren interne Repr�sentation}

F�r die Beschreibung der XML-Dateistruktur siehe Sektion \ref{sec:benutzerhandbuch:anleitung}. Intern wird ebendiese Struktur �bernommen.

\mylisting[label=lst:output, caption={Interne Darstellung der Konfiguration}]{listings/output.h}

\subsection{Formatierung der Bilddateien}
Die Bilddateien, die optional in der Konfigurationsdatei angegeben werden k�nnen sind "`TGA"' Dateien (Truevision Graphics Adapter). Auch bezeichnet als "`TARGA"'. Die Bilddateien m�ssen um korrekt gelesen zu werden unkomprimierte Daten ohne Farbtabelle enthalten. Dabei werden vier Byte pro Pixel abgelegt, je ein Byte f�r rot, gr�n, blau und alpha Darstellung. Wesentlich ist aktuell nur der Alpha-Wert, der beschreibt wie stark der jeweilige Ausgabebereich abgedunkelt wird.