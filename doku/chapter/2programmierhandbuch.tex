%%%%%%%%%%%%%%%%%%%%%%%%%%%%%%%%%%%%%%%%%%%%%%%%%%%%%%%%%%%%%%%%%%%%%%%
\chapter{Programmierhandbuch}
\label{cha:programmierhandbuch}

%Es gilt der Grundsatz der Transparenz: Ein Programm mu� von einem sachverst�ndigen Dritten in "angemessener Zeit" �berblickt und gepr�ft werden k�nnen. Dieser Teil der Dokumentation soll also einem Programmierer einen �bersichtlichen Einblick in das Programm geben und somit die Wartung erm�glichen. Dazu geh�ren folgende Dokumentationspunkte:

Dieses Handbuch gibt einen �berblick �ber die Entwicklungkonfiguration der zu Entwicklung benutzten Systeme, die grundlegenden Datenstrukturen und die Programmorganisation. Der Hauptteil besteht aus der Problemanalyse und -realisation, in der dem Leser ein Einblick in die Gedanken hinter der Umsetzung des Plug-In gegeben wird.