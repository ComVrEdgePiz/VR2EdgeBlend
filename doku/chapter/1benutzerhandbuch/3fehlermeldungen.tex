%%%%%%%%%%%%%%%%%%%%%%%%%%%%%%%%%%%%%%%%%%%%%%%%%%%%%%%%%%%%%%%%%%%%%%%
\section{Fehlermeldungen}
\label{sec:benutzerhandbuch:fehler}

%Welche Fehlermeldungen gibt es? Wie ist darauf zu reagieren? Eine Aufbereitung dieser Informationen in Form einer Tabelle ist am kompaktesten (Meldung / Bedeutung / Ma�nahme).
Fehler treten in der Regel nur durch Eingaben auf, diese sind bei dem Edgeblend Plug-In auf die XML Konfigurationsdatei und die optionale Datei aus dem \texttt{image}-Tag beschr�nkt. Ist eine dieser Dateien, wenn ben�tigt, nicht vorhanden oder fehlerhaft formatiert l�sst sich das Plug-In nicht starten. Im Zweifelsfall ist die Bilddatei zu l�schen und die XML Konfigurationsdatei auf korrektheut zu �berpr�fen.
%- fehler (xml nicht gefunden) (siehe fehlermeldungen)
% - genauere fehlerinformationen wenn auf der console (compiz) gestartet wird

Danach ist gegebenenfalls das Plug-In neu zu installieren und erneut zu starten.