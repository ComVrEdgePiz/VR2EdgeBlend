%%%%%%%%%%%%%%%%%%%%%%%%%%%%%%%%%%%%%%%%%%%%%%%%%%%%%%%%%%%%%%%%%%%%%%%
\section{Ablaufbedingungen}
\label{sec:benutzerhandbuch:ablauf}

In dieser Sektion werden die Hard- und Softwarevoraussetzungen beschrieben, die zur erfolgreichen Benutzung des Plug-In notwendig sind.
%Dieser Abschnitt enth�lt Informationen dar�ber, welche Voraussetzungen f�r den Programmstart erforderlich sind, welche Hardware (z.B. Grafikkarte, Monitor) und Software (in welcher Version) erforderlich ist, welche Dateien in welchen Verzeichnissen vorhanden sein m�ssen etc. Es ist zu bedenken, da� die Ablaufbedingungen normalerweise von der Entwicklungskonfiguration abweichen.

\subsection{Hardware}
Das Plug-In ben�tigt haupts�chlich zum erzeugen der zu �berblendenden Grafik CPU resourcen, bei h�herer Prozessorleistung startet es somit schneller. Ansonsten wird eine OpenGL f�hige Grafikkarte mit mindestens 64 MB Grafikspeicher ben�tigt, sowie mindestens zwei Anzeigeger�te mit gleicher (eingestellter) vertikaler und horizontaler Aufl�sung

\subsection{Software}
Das Plug-In ist nur auf einem Linux-System mit X-Server lauff�hig, da \emph{Compiz} (compiz.org) und X-Server (X.org) Methoden verwendet werden. Die Installation von \emph{Compiz} ist in Sektion \ref{sec:benutzerhandbuch:install_start} beschrieben.

