%%%%%%%%%%%%%%%%%%%%%%%%%%%%%%%%%%%%%%%%%%%%%%%%%%%%%%%%%%%%%%%%%%%%%%%
\section{Bedienungsanleitung}
\label{sec:benutzerhandbuch:anleitung}

%An diese Stelle geh�rt eine Beschreibung der angebotenen Funktionen und wie sie aktiviert werden, also wie der Benutzer vorgehen mu�, um seine Probleme zu l�sen bzw. mit dem Programm zu arbeiten. Zur Verdeutlichung und Orientierung ist es auch sinnvoll, an dieser Stelle Screenshots der Benutzeroberfl�che einzuf�gen. Die Ein- und Ausgabedaten (Inhalte und Formate) sind zu beschreiben.

Da nach dem erfolgreichen Start des Plug-In nichts mehr einstellbar ist, werden hier die Konfigurationsm�glichkeiten, die �ber die XML-Formatierte Datei zur Verf�gung stehen, beschrieben.

Das Edgeblend Plug-In kann XML-Tags verarbeiten die in der folgenden Struktur auftreten m�ssen:
\begin{inlinexml}
output
- grid
- - rows, cols
- - cell
- - - height, width
- screens
- - screen
- - - top, left, right, bottom
- - - - a, b, c
- - image
\end{inlinexml}

Alle Attribute der Tags werden ignoriert, es werden nur die Tags und der Inhalt dieser ausgewertet.

\begin{description}
  \item [output] der Root-Tag f�r valides XML
  \item [grid] allgemeine Einstellungen, die f�r die gesamte Sichtfl�che gelten
  \item [rows] Anzahl der vertikal angeordneten grafischen Ausgabeger�te
  \item [cols] Anzahl der horizontal angeordneten grafischen Ausgabeger�te
  \item [cell] Aufl�sung eines Ausgabeger�tes (muss f�r jedes gleich sein)
  \item [height] vertikale Aufl�sung eine Ausgabeger�tes
  \item [width] horizontale Aufl�sung eines Ausgabeger�tes
  \item [screens] Einstellungen f�r die Ausgabeger�te
  \item [screen] Blending-Einstellungen f�r einen Ausgabeger�t (die Anzahl der \texttt{screen}-Tags muss mit der Anzahl der \texttt{rows} $*$ \texttt{cols} �bereinstimmen, die Reihen der Konfigurationen l�uft von von oben links nach unten rechts)
  \item [top, left, right, bottom] Blending-Einstellung f�r eine Seite eines \texttt{screen}
  \item [a, b, c] Parameter f�r die Blending-Funktion einer Seite eines Ausgabeger�tes.
  \item [image] Pfad zu einer Bilddatei, die die Aufl�sung aller Ausgabeger�te zusammen enth�lt (Alternative zu den \texttt{screen}-Tags)
\end{description}

\mylisting[label=lst:config, caption={Eine Beispielkonfiguration}]{listings/config.xml}

Diese Konfigurationsdatei Verwendet vier im quadrat angeordneten Ausgabeger�te mit je einer Aufl�sung von 1024x768. Die Einzelkonfiguration der Ausgabeger�te wird von oben links zeilenweise nach unten rechts gelesen. Wenn die Datei unter dem \texttt{image} Pfad existiert wird diese geladen, ansonsten werden die Einzelkonfigurationen verwendet. Hier wird zwischen dem Ger�t oben links zu dem oben rechts beginnend ab 20 Pixel vom Rand entfernt �bergeblendet. Von oben nach unten wird links, sowie rechts mit 30 Pixel �bergeblendet und unten zwischen dem linken und rechten Ausgabeger�t je mit 40 Pixel, so dass sich eine Sichtfl�che von 2028 Pixel breite im oberen Bereich und 2008 Pixel breite im unteren Bereich bildet. Die H�he betr�gt dabei 1506 Pixel �ber die gesamte Breite.